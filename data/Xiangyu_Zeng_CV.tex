\documentclass[11pt,a4paper]{article}
\usepackage[margin=0.75in]{geometry}
\usepackage{enumitem}
\usepackage{titlesec}
\usepackage{hyperref}

% Remove page numbers
\pagestyle{empty}

% Section formatting
\titleformat{\section}{\large\bfseries\uppercase}{}{0em}{}[\titlerule]
\titlespacing{\section}{0pt}{12pt}{8pt}

% Remove paragraph indentation
\setlength{\parindent}{0pt}

% Adjust item spacing
\setlist{leftmargin=*, nosep, itemsep=3pt, topsep=3pt}

% Vertical spacing between paragraphs
\setlength{\parskip}{3pt}

\begin{document}

% Header
\begin{center}
    {\LARGE \textbf{Xiangyu Zeng}} \\[6pt]
    Tel: +12177213642 \quad Email: xzeng28@illinois.edu
\end{center}

\vspace{4pt}

\section{Research Interests}

Vision-Language-Action Models for Robot Manipulation $\cdot$ Imitation Learning and Diffusion Policy $\cdot$ Transformer-based Policy Learning $\cdot$ Trajectory Prediction and Motion Planning $\cdot$ Deep Learning for Autonomous Systems

\section{Education}

\textbf{University of Illinois Urbana Champaign} \hfill 09/2025-Present
\begin{itemize}
    \item MEng in Mechanical Engineering, concentration in Control \& Robotics
\end{itemize}

\vspace{2pt}

\textbf{Wuhan University of Technology} \hfill 09/2021-06/2025
\begin{itemize}
    \item Bachelor of Engineering in Intelligent Manufacturing Engineering \hfill GPA: 4.071/5.0
\end{itemize}

\section{Honors \& Awards}

\begin{itemize}
    \item First Prize of Yanchang Petroleum Scholarship (Top 0.5\%), Wuhan University of Technology, 2023
    \item University-level First Class Scholarship (Top 3\%), Wuhan University of Technology, 2022-2024
    \item Outstanding Student Class Leader, Wuhan University of Technology, 2023
    \item Independent Innovation Research Fund (Undergraduate Program), Wuhan University of Technology, 2022
\end{itemize}

\section{Publications}

\vspace{-2pt}

[1] Yiying Wei*, \textbf{Xiangyu Zeng}*(co-first author), Xirui Chen. HTSA-LSTM: Leveraging Driving Habits for Enhanced Long-Term Urban Traffic Trajectory Prediction. Accepted by \textit{Applied Sciences}.

\vspace{3pt}

[2] Xu F, Zong W, Zeng X, et al. Construct a health indicator for bearing based on unsupervised SDAE with Euclidean distance[J]. \textit{Journal of Vibration and Control}, 2025: 10775463251342614

\vspace{3pt}

[3] \textbf{Xiangyu Zeng}, Fan Xu. Construction of Bearing Degradation Health Indicators Based on Unsupervised Augmented Noise Reduction Deep Belief Network. Submitted to \textit{Quality and Reliability Engineering International}. Under Review.

\vspace{3pt}

[4] \textbf{Xiangyu Zeng}, Zhou Hang, Lei Wang. Research on Batch Scheduling of Unrelated Parallel Machines with Flexible Job Sequences for the Standard Gas Manufacturing Sector. Submitted to \textit{Computers \& Industrial Engineering}. Under Review.

\section{Research Experiences}

\vspace{-2pt}

\textbf{Research on Enhanced Long-Term Urban Traffic Trajectory Prediction Model (HTSA-LSTM)} \\
\textit{Supervised by Dr. Yiying Wei from Wuhan University of Technology} \hfill 06/2024-06/2025
\begin{itemize}
    \item Designed HTSA-LSTM network integrating dual spatiotemporal attention mechanism with driving style analysis for autonomous vehicle trajectory prediction
    \item Developed a novel driving style analysis module using SICC-SC to extract driving primitives and cluster trajectory patterns without predefined labels
    \item Implemented spatiotemporal attention mechanisms to capture dynamic dependencies across time and space
    \item Evaluated on the NGSIM dataset with significant improvements over benchmark models: 20.72\% reduction in RMSE and 24.98\% reduction in NLL for 5-second predictions
    \item Carried out real-world validations on two types of roads in Wuhan, achieving: R\textsuperscript{2}>0.979 for 5-second highway predictions and R\textsuperscript{2}>0.927 for 3-second urban road predictions
\end{itemize}

\vspace{3pt}

\textbf{Construction of Bearing Degradation Health Indicators Based on Unsupervised Augmented Noise Reduction Deep Belief Network} \hfill 11/2022-03/2024 \\
\textit{Team Leader, Wuhan University of Technology Independent Innovation Research Fund Undergraduate Program, Supervised by Prof. Fan Xu from Wuhan University of Technology}
\begin{itemize}
    \item Gathered and preprocessed the bearing data provided by the University of Cincinnati lab
    \item Constructed an SGDBN model using Python, which is processed with Savitzky-Golay filters between each RBM layer of the DBN and after the output layer, to process the data generated during bearing operation and reduce noise, thus accurately predicting bearing life
    \item Examined and selected the optimal filter parameters, learning rate, and neural network structure in turn through multiple sets of experiments
    \item Assessed the performance of the SGDBN model on the test set using the Mon smoothing metrics, finding that the Mon value obtained reaches 7-10 times that obtained by traditional RMS and K-Mediods, and 3-5 times that obtained by SOM and DBN, proving it extracts health indicators with the highest smoothing and noise reduction performance
    \item Demonstrated the excellent performance of this model in constructing bearing degradation curves and determining the health status of bearings for fault diagnosis
\end{itemize}

\vspace{3pt}

\textbf{Non-Standard Gas Filling Plant Scheduling Problem Based on Imperialist Competitive Algorithm} \\
\textit{Supervised by Prof. Lei Wang from Wuhan University of Technology} \hfill 11/2022-04/2024
\begin{itemize}
    \item Developed a Federated Imperialist Competitive Algorithm (FICA) using MATLAB to address the scheduling challenges in a non-standard gas filling plant, overcoming the local optima issues of traditional ICA through federated partitioning and local optimization
    \item Constructed a mathematical model to minimize makespan and resource consumption, considering the complexity of flexible filling, batch processing, and heterogeneous equipment
    \item Introduced a dual-layer optimization strategy for both intra- and inter-batch scheduling, significantly enhancing search efficiency and solution quality
    \item Demonstrated that FICA improves scheduling performance, reducing processing time by approximately 15\% and machining losses by 20\% compared to traditional ICA, and both by over 30\% compared to the genetic algorithm
\end{itemize}

\section{Technical Skills}

\vspace{-2pt}

\textbf{Programming Languages:} Python, MATLAB \\[2pt]
\textbf{ML/DL Frameworks:} PyTorch \\[2pt]
\textbf{Robotics Platforms:} Kinova Gen3, UR5 \\[2pt]
\textbf{Simulation \& Tools:} ROS/ROS2, Gazebo, Isaac Sim/Isaac Gym \\[2pt]
\textbf{Methods:} Imitation Learning, Diffusion Policy, Transformer-based Policy Learning

\end{document}
